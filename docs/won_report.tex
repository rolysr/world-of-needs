\documentclass[12pt]{amsart}

\usepackage{amsmath}
\usepackage{amsfonts}
\usepackage{amssymb}
\usepackage[utf8]{inputenc}
\usepackage{listings}
\usepackage[pdftex]{hyperref}
\usepackage{caption}
\usepackage{subcaption}
\usepackage{geometry}
\geometry{
 letterpaper,
 total={8.5in,11in},
left=25mm,
right=25mm, 
top=20mm,
bottom=20mm,
}
 \usepackage{setspace}
\doublespace 
\usepackage[document]{ragged2e}

\usepackage{helvet}
\renewcommand{\familydefault}{\sfdefault}

\usepackage{blindtext}

\usepackage[spanish]{babel}

\usepackage{graphicx}
\usepackage{epsfig}

\begin{document}

\begin{center}
    {\large Universidad de La Habana}  \\ 
    \vskip 0.1cm
    {\LARGE \textbf{T\'ITULO DEL TRABAJO}} \\
    \vskip 2cm
    {\Large Autores:}\\ 
    \vspace{0.5cm} 
    		{\Large\textbf{AUTOR1}} \\
    		{\normalsize\textit{autor1@estudiantes.matcom.uh.cu}}\\
			FACULTAD\\
			\smallskip
 			{\Large\textbf{AUTOR2}}\\
 			{\normalsize\textit{autor2@estudiantes.matcom.uh.cu}}\\
 			FACULTAD \\
 \smallskip
 \smallskip
    {\Large Tutores:\\ 
    \vspace{0.05cm} 
    		\textbf{TUTOR1\\ TUTOR2}} \\    
  \vskip 1.5cm
  
  \Large \textbf{Resumen}:
    Resumen del trabajo (Extensi\'on m\'axima 250 palabras)
  \end{center}
  


\newpage
\section{Introducci\'on}

\paragraph{Se expondr\'an los antecedentes del trabajo y se precisar\'an claramente sus objetivos.}

\section{Desarrollo}

\paragraph{Incluir\'a los aspectos metodol\'ogicos y los resultados obtenidos.}

	\subsection{Listas y Descripciones}\label{sub:lists}
		Para producir listas enumeradas, utilice el siguiente estilo:
		\begin{enumerate}
			\item Primer Elemento
			\item Segundo Elemento
			\begin {enumerate}
				\item {Segundo Elemento - Subítem Uno}
				\item {Segundo Elemento - Subítem Dos}
			\end {enumerate}
		\end{enumerate}

		Para producir descripciones, use el siguiente estilo:

		\begin{description}
			\item [Primer Elemento] con su respectiva descripción.
			\item [Segundo Elemento] también con su respectiva descripción.
		\end{description}

	\subsection{Figuras}\label{sub:figures}
		Para producir cuerpos flotantes (figuras o tablas), asegúrese de numerar
		y etiquetar correctamente cada figura. Las referencias a las figuras deben
		estar correctamente etiquetadas. Por ejemplo, véase la Fig. \ref{fig:ex}\ldots

		\begin{figure}[h!]%
		\begin{center}
			\begin{tabular}{|c|c|c|} \hline
			 			& Método 1 	& Método 2 	\\ \hline
			A 			&  			&  			\\ \hline
			B			& 			& 			\\ \hline
			C 			& 			&  			\\ \hline
			\end{tabular}
		\caption{Figura de ejemplo. Recuerde especificar el origen de los datos que se muestran. \label{fig:ex}}
		\end{center}
		\end{figure}

\section{Conclusiones}

\paragraph{Se refiere al cumplimiento de los objetivos}

\section{Recomendaciones}

\begin{thebibliography}{99}
	\bibitem{knuth} Donald E. Knuth. \emph{The Art of Computer Programming}.
		Volume 1: Fundamental Algorithms (3rd~edition), 1997.
		Addison-Wesley Professional.

	\bibitem{goedel} Kurt Göedel. \emph{Über formal unentscheidbare Sätze der
		Principia Mathematica und verwandter Systeme, I}.
		Monatshefte für Mathematik und Physik 38.

	\bibitem{wiki} Wikipedia. URL: \href{http://en.wikipedia.org}
	  {http://en.wikipedia.org}.
		Consultado en \today.
\end{thebibliography}
\smallskip
\smallskip
\textbf{Importante: El artículo debe tener una extensión máxima de 15 páginas, sin contar la portada.}
\end{document}